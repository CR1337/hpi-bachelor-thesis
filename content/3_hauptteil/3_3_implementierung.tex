\section{Implementierungsdetails}
\todo{Deadline: 27.06.}
\todo{Noch in Arbeit}
\todo{Hier zeige ich Code, der interessant sein könnte.}
\todo{FRAGE: Ist es ok, wenn ich nur ''interessanten'' Code zeige? Ansonsten bin ich der Meinung, dass die Sequenzdiagramme und der Text in der Architekturdiskussion das implementierte Verhalten bereits gut beschreibt.}

\subsection{Warteschlange für die Zugerzeugung}

\lstset{language=python}
\begin{lstlisting}[caption={Quelltextunterschrift}, label=code:template-method-code]
class Schedule(ABC):

    _seconds_to_be_spawned: list[int]
    _blocked: bool

    def maybe_spawn(self, seconds: int, spawner: Spawner):
        if not self._blocked and self.strategy.should_spawn(seconds):
            self._seconds_to_be_spawned.append(seconds)

        if len(self._seconds_to_be_spawned) > 0:
            if self._spawn(spawner, self._seconds_to_be_spawned[-1]):
                self._seconds_to_be_spawned.pop()

    @abstractmethod
    def _spawn(self, spawner: Spawner, seconds: int) -> bool:
    	raise NotImplementedError()
\end{lstlisting}

\subsection{Erzeugung bedarfsorientierter Züge}

\lstset{language=python}
\begin{lstlisting}[caption={Quelltextunterschrift}, label=code:template-method-code]
class DemandScheduleStrategy(ScheduleStrategy):
    SECONDS_PER_QUARTER_HOUR: int
    COAL_PER_TRAIN: float

    start_datetime: datetime
    spawn_seconds: list[int]

    def __init__(self, ...):
        ...
        self._calculate_spawn_seconds()
        ...

    def _compute_trains_to_spawn(self, produced_electrical_power: float) -> float:
        coal_consumption = self.compute_coal_consumption(produced_electrical_power)
        return coal_consumption / self.COAL_PER_TRAIN

    def _calculate_spawn_seconds(self):
        end_datetime = self.start_datetime + timedelta(
            seconds=self.end_time - self.start_time
        )
        data = self._api.get_data(self.start_datetime, end_datetime)
        train_accumulator = 0.0
        for quarter_hour, entry in enumerate(data):
            train_accumulator += self._compute_trains_to_spawn(entry.value)
            seconds = int(quarter_hour * self.SECONDS_PER_QUARTER_HOUR + self.start_time)
            while train_accumulator >= 1.0:
                self.spawn_ticks.append(seconds)
                train_accumulator -= 1.0
                tick += 1

	def should_spawn(self, seconds: int) -> bool:
    	return super().should_spawn(seconds) and seconds in self.spawn_seconds
\end{lstlisting}