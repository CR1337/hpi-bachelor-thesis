\section{Ergebnisdiskussion}

Die Ergebnisse in Abbildung \ref{fig:results-departures} zeigen, dass die Längen der Abfahrtsperioden der Kohlezüge direkt mit dem berechneten Kohlebedarf korrelieren. Das bedeutet, dass umso mehr Kohlezüge abfahren, je mehr Kohle benötigt wird. Dieses Ergebnis ist sinnvoll und deckt sich mit der Realität und dem was man erwarten würde. Laut Sascha Lesche (siehe Anhang Eins) können an einem Tag etwa 80 Züge in ein großes Kraftwerk einfahren. Diese Zahl kann jedoch je nach tatsächlichem Bedarf auch höher oder niedriger sein. Umgerechnet ergibt sich aus diesem Wert eine mittlere Abfahrtsperiodenlänge von 18 Minuten. Das Ergebnis für das größte der drei Kraftwerke, \emph{Jänschwalde}, liegt mit eine mittleren Abfahrtsperiodenlänge von etwa 44 Minuten etwa bei dem 2,5-fachen dieses Wertes. Dies kann daran liegen, dass das Kraftwerk im betrachteten Zeitraum nicht vollständig ausgelastet war. Wahrscheinlicher ist jedoch, dass die für die Berechnung getroffenen Annahmen der Grund für diese Abweichung sind.\\
\\
Die mittlere Abfahrtsperiodenlänge der drei Kraftwerke \emph{Jänschwalde}, \emph{Boxberg} und \emph{Schwarze Pumpe} deckt sich mit der Realität. Das Kraftwerk \emph{Jänschwalde} hat mit einer Gesamtleistung von run 3,3 GW die höchste der drei Kraftwerke. Entsprechend muss es den höchsten Kohlebedarf besitzen, was die niedrigste Abfahrtsperiodenlänge von ca. 44 Minuten plausibel macht. Gleiches gilt für das zweit-leistungsfähigste Kraftwerk \emph{Boxberg} mit einer Gesamtleistung von eta 2,7 GW und auch für das Kraftwerk\emph{Schwarze Pumpe} mit ca. 1,7 GW. \cite{noauthor_bundesnetzagentur_nodate}\\
\\
Zusammenfassend lässt sich sagen, dass die berechneten Werte durchaus Abweichungen von der Realität zeigen, sie jedoch in der selben Größenordung liegen. Dies liegt aller Wahrscheinlichkeit nach in den getroffenen Annhamen und Vereinfachungen begründet. Am Beispiel des Kraftwerkes \emph{Jänschwalde}, bei welchem die Abweichung einen Faktor von 2,5 aufweist, ließe sich diese Ungenauigkeit durch die Verwendung eines Skalierungsfaktors von 2,5 beheben. Für einen solchen Skalierungsfaktor ist das konfigurierbare Attribut \code{scaling\_factor} in der Klasse \code{DemandScheduleStrategy} vorgesehen. Abweichungen von der Realität lassen sich also damit korrigieren.