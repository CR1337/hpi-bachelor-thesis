\section{Die Smard-API}

Die Smard-API\cite{noauthor_bundesapismard-api_nodate} ist eine von der Bundesnetzagentur bereitgestellte Schnittstelle zum Abrufen von Stromproduktionsdaten in Deutschland. Sie liefert die Menge produzierten Stromes in Abhängigkeit von der Zeit. Die Daten werden in verschiedenen zeitlichen Auflösungen, Jahresdaten bis hin zu 15 Minuten-Intervallen, bereitgestellt. Räumlich lassen sich die Daten bis auf eine Regelzone\footnote{''Als Regelzone wird ein räumlich abgegrenztes Netzgebiet bezeichnet für das ein Übertragungsnetzbetreiber verantwortlich ist.''\cite{noauthor_smard_nodate}} eingrenzen. Weiterhin lassen sich die Daten nach dem Energieträger filtern. So besteht die Möglichkeit, Daten speziell für Braunkohle zu erhalten.\\
\\
Bei der Smard API handelt es sich um eine statische API. Das bedeutet, die Antwort auf eine Anfrage wird nicht für jeden Benutzer zusammengestellt. Stattdessen existiert eine Menge von JSON-Dateien, von denen je nach Anfrage eine ausgeliefert wird. Die API stellt zwei Arten von Dateien zur Verfügung. Für jede Kombination aus zeitlicher und räumlicher Auflösung und Filter gibt es eine Index-Datei. Diese beinhaltet eine Liste von Zeitstempeln. Jeder Zeitstempel ist mit einer Daten-Datei assoziiert. Durch Verwendung des Zeitstempels kann die zugehörige Daten-Datei heruntergeladen werden. Diese enthält eine Liste von Paaren aus Zeitpunkt und produzierter Energiemenge in MWh. Die Paare sind aufsteigend nach Zeitpunkten sortiert und der erste Zeitpunkt entspricht dem verwendeten Zeitstempel.