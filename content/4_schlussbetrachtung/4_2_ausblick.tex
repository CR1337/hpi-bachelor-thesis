\section{Ausblick}

Die von uns entwickelte Software stellt eine Erweiterung des Verkehrssimulators SUMO dar, welche es ermöglicht, Zugverkehr auf einem realistischeren Niveau durchzuführen, als dies vorher möglich war. Wir stellen eine Plattform bereit, welche Züge mithilfe einer Stellwerkslogik durch ein Schienennetz bewegen kann. Für die Abfahrt der Züge lassen sich sogenannte Abfahrspläne konfigurieren. Damit können bisher regelmäßige und zufällige Abfahrten simuliert werden, als auch solche, die sich aus dem Kohlebedarf eines Krafterkes ergeben. Insbesondere die letzte Möglichkeit ist von besonderer Bedeutung für die Fragestellungen, die es im Rahmen des Projektes \emph{FlexiDug} zu beantworten gilt. Natürlich gibt es nach wie vor Aspekte des Bahnverkehrs, welche nicht durch unser System abgebildet werden können.\\
\\
Eine Möglichkeit zur Erweiterung der Software, um den Realismus der Simulation zu erhöhen, sind \emph{Reaktive Abfahrtspläne}. Diese würden es ermöglichen, dass Züge auf Ereignisse in der Simulation reagieren. Zentrales Element der Implementierung dieser Abfahrtspläne ist der \emph{EventBus}, welcher in der Software bereits implementiert ist. Die Routenberechnung könnte Beispielsweise ein Ereignis auslösen, wenn ein Zug seinen Endhaltestelle erreicht hat. Ein reaktiver Abfahrtsplan könnte dann auf dieses Ereignis reagieren und einen neuen Zug abfahren lassen. Damit ließe sich die Rückfahrt eines Zuges simulieren. Auch können damit Abhängigkeiten zwischen Abfahrten modelliert werden. So könnte ein Zug erst abfahren, wenn ein anderer Zug eine bestimmte Haltestelle erreicht hat, beispielsweise weil das Zugpersonal selbst erst mit der Bahn anreisen muss. Eine weitere zukünftige Verwendungsmöglichkeit des Eventbus wird von \citeauthor{persitzky_fehlerinjektion_2023} in seiner Arbeit\cite{persitzky_fehlerinjektion_2023} diskutiert.\\
\\
Es liegt nun in der Hand der Projektpartner von \emph{FlexiDug}, ihre Fragestellungen mithilfe unseres Systems zu untersuchen. Sie verfügen damit über ein Werkzeug, welches es ihnen ermöglicht, die Auswirkungen verschiedener Szenarien für den Bahnverkehr im Schienennetz der LEAG zu untersuchen. Aufgrund der Flexibilität und Erweiterbarkeit der Software, kann diese in Zukunft auch für andere Schienennetze und ähnliche Fragestellungen eingesetzt werden.
