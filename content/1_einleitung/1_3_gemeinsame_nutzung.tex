\section{FlexiDug - Gemeinsam Nutzung durch Personen- und Güterzüge}

Nach dem Kohleausstieg würde das Potential des Schienennetzes ohne ein Nachnutzungskonzept ungenutzt bleiben\cite{rbb_hpi_2022}. Das Projekt \emph{FlexiDug} (Flexible, digitale Systeme für den schienengebundenen Verkehr in Wachstumsregionen) beschäftigt sich damit, Nachnutzungsperspektiven zu erstellen und ein solches Konzept bis zum Jahr 2024 zu erstellen \cite{hasso_plattner_institut_flexidug_2022}. Teil dieses Projektes sind die Analyse von Infrastrukturen, die Erstellung eines digitalen Zwilling des Schienennetzes, die Entwicklung von Sensornetzen zur Infrastrukturüberwachung und die Erforschung digitaler Leit- und Sicherungstechnik. All dies würde erst einen Personenverkehr auf diesem Schienennetz ermöglichen. Das große Ziel des Projektes ist der ''Ergründen von Perspektiven'' für die Nachnutzung der Infrastruktur und die Erforschung der Möglichkeit einer sowie einer Nachnutzung, als auch einer gemeinsamen Nutzung durch Personen- und Güterzüge.\cite{rbb_hpi_2022} Diese Arbeit im Kontext des gesamten Softwareprojektes leistet einen Beitrag durch die Erweiterung des Verkehrssimulators SUMO um ein realistischeres Modell des Zugverkehrs. Damit wird die Simulation von gleichzeitig fahrenden Personen- und Güterzügen auf dem Schienennetz der LEAG ermöglicht.
