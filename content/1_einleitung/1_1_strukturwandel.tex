\section{Strukturwandel in der Lausitz}

Die Lausitz, eine Region in Südbrandenburg, steht in naher Zukunft vor tiefgreifenden Veränderungen. Mit rund 8000~Beschäftigten \cite{noauthor_braunkohle_nodate} ist der Braunkohleabbau durch die LEAG AG ein bedeutender Arbeitgeber in der Region. Braunkohle ist ein wichtiger Energieträger für die Strom- und Fernwärmeproduktion in der Lausitz. Im Rahmen der Energiewende soll jedoch die Energieproduktion aus Braunkohle bis zum Jahr 2038 zugunsten von erneuerbaren Energiequellen vollständig eingestellt werden. Dies wird die Region (und auch ganz Deutschland) vor eine Vielzahl von finanziellen, wirtschaftlichen und gesellschaftlichen Herausforderungen stellen. Es müssen neue Wirtschaftskonzepte für die Region entwickelt werden, um diesen Strukturwandel zu bewältigen. Dazu zählt zum einen eine Alternative für den Bergbau zu finden, um den vorhandenen Arbeitnehmern weiterhin Arbeitsplätze zur Verfügung stellen zu können. Zum anderen soll der Tourismus in der Region stark, durch die Nachnutzung der entstehenden Flächen als Erholungsgebiet, ausgebaut werden. Um das erreichen zu können, ist es notwendig, die Tagebaulandschaft zu renaturieren \cite{btu_flexidug_2022} und die notwendigen Flächen und Infrastrukturen zu erschließen. Dieser Strukturwandel stellt für die Lausitz den größten Wandel seit dem Strukturbruch im Jahr 1990 im Rahmen der Wiedervereinigung dar.

Es ist daher notwendig, die entsprechenden Maßnahmen ausreichend zu planen und mit den jeweiligen Interessenvertretern abzustimmen. Mit einem Teil dieser Planung beschäftigt sich das Projekt \emph{FlexiDug}, welches in den folgenden Abschnitten genauer beschrieben wird. Es erkundet Nachnutzungsmöglichkeiten der Schieneninfrastruktur, welche bisher ausschließlich dem Transport der Braunkohle diente und aktuell in privater Hand liegt.