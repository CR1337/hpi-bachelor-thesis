\section{Object-Relational-Mapping}

\subsection{Objekt-relationale Systeme}
In der Praxis haben sich für die Beschreibung von Datenbanken und Software vor allem zwei verschiedene Paradigmen etabliert. Datenbanken, welche der persistenten Speicherung von Daten dienen, folgen häufig dem relationalen Paradigma. Softwaresysteme hingegen, deren Verhalten und Struktur durch Programmiersprachen beschrieben werden, werden meist objektorientiert modelliert. Da diese beiden Technologien, breite Verwendung finden, ist ihre Kombination innerhalb desselben Systems in der häufig unumgänglich. Solche Systeme nennt man objekt-relationale Systeme. \cite{ireland_understanding_2009}

\subsection{Object-Relational Mapper (ORM)}
Die Aufgabe eines ORM ist es, die Persistenz von Objekten zu gewährleisten \cite{noauthor_what_2023} und dabei gleichzeitig den Programmierer von der darunterliegenden relationalen Datenbank abzuschirmen. So kann der Programmierer, welcher eine objektorientierte Programmiersprache verwendet, vollständig in einem objektorientierten Kontext arbeiten. Er ist nicht mehr gezwungen, direkt mit der relationalen Datenbank über SQL zu kommunizieren. Das ORM muss dabei eine bidirektionale Abbildung zwischen dem objektorientierten und dem relationalen Modell bereitstellen. Dabei muss es sicherstellen, dass sowohl die Struktur, als auch die Mechanismen beider Paradigmen korrekt aufeinander abgebildet werden. Die Abbildung der Struktur beschäftigt sich mit der Zuordnung von Tabellen zu Klassen. Die Abbildung der Mechanismen behandelt unter anderem die Navigation durch Objektreferenzen und das Schreiben und Lesen von Daten. \cite{ireland_understanding_2009}

\subsection{Hindernisse}
Die Unterschiede zwischen beiden Paradigmen rufen eine Reihe von Problemen hervor, welche das ORM lösen muss. Diese Probleme sind nach \cite{ireland_understanding_2009} die folgenden:
\begin{itemize}
    \item Es muss eine Abbildung zwischen den Strukturen beider Paradigmen gefunden werden. Eine relationale Datenbank unterstützt weder Klassenhierarchien noch Spalten, die mehrere Elemente beinhalten können.
    \item Zeilen innerhalb einer Relation haben eine festgeschriebene Struktur. Objekte hingegen können eine dynamische Struktur besitzen. Die Frage ist, wie sich die Objektstruktur in der Datenbank abbilden lässt.
    \item Während der Zustand eines Objektes durch Kapselung geschützt ist, sind alle Daten einer relationalen Datenbank öffentlich. Das Problem besteht in der Abbildung dieser Kapselung.
    \item ''Identität'' hat innerhalb der beiden Paradigmen unterschiedliche Bedeutungen. Objekte gelten als identisch, wenn sie die gleiche Speicheradresse besitzen. Identische Zeilen hingegen zeichnen sich durch einen gleichen Primärschlüssel aus. Das kann zu Problemen führen, wenn zwei nicht-identische Objekte mit gleichem Primärschlüssel erzeugt werden.
    \item Referenzen in objektorientierten und relationalen Modellen besitzen unterschiedliche Richtungen. Entsprechend muss die Navigation durch die Modelle abgebildet werden.
    \item In der Praxis kann es vorkommen, dass die Datenbank und die darauf aufbauende Software von unterschiedlichen Teams gepflegt werden. Außerdem besteht die Möglichkeit, dass eine Datenbank von mehreren Softwaresystemen genutzt wird. Dabei gilt es, eine einheitliche Kommunikation zu gewährleisten.
\end{itemize}
Kein ORM löst alle diese Probleme vollständig. Jedes konzentiert sich auf die Aspekte der zu leistenden Abbildung unterschiedlich stark. So kann der Fokus beispielweise mehr auf der strukturellen Abbildung zwischen Klassen und Tabellen liegen oder auf der korrekten Abbildung der Mechanismen. Es gibt daher keine einheitliche oder richtige Lösung. Entsprechend ist die Trennung zwischen objektorientiertem und relationalem Paradigma nie vollständig und hängt vom Einzelfall und den Bedürfnissen des Anwenders ab.

\subsection{\emph{peewee}}
Für unser Softwareprojekt haben wir das ORM \emph{peewee} \cite{leifer_coleiferpeewee_2023} ausgewählt. Da das ORM einer der Grundsteine unseres Projektes darstellt, stand die Auswahl des ORM am Anfang Planungsprozesses. Daher war es uns nicht möglich, das ORM  anhand der Anforderungen unserer Softwarearchitektur auszuwählen. Diese stand zu diesem Zeitpunkt, aufgrund der agilen Arbeitsweise unseres Teams, noch nicht in ausreichendem Detailgrad zur Verfügung. Wir haben die Entscheidung stattdessen anhand der allgemeinen Funktionsübersicht und der Benutzerfreundlichkeit getroffen. Zur Auswahl stand außerdem das ORM ''SQLAlchemy'' \cite{noauthor_sqlalchemy_2023}. Obwohl SQLAlchemy einen größeren Funktionsumfang bietet, ist \emph{peewee} einfacher zu bedienen und man erreicht viele der grundlegenden Funktionen mit weniger Code. Da dies zugunsten der Entwicklungsgeschwindigkeit geht, haben wir uns letztendlich für \emph{peewee} entschieden. Außerdem sinkt die Wahrscheinlichkeit für das Machen von Fehlern mit weniger Code.

Im folgenden wird eine Reihe von Grundfunktionalitäten von \emph{peewee} vorgestellt, welche häufig in unserem Projekt Verwendung fanden und die auch im Rahmen dieser Arbeit eine Rolle spielen.

\subsubsection*{Definition von Modellen}

Als ein Modell wird im Kontext von \emph{peewee} eine Klasse bezeichnet, deren Objekte sich in der verknüpften Datenbank ablegen lassen. Wie in Quelltext \ref{code:peewee-model} zu erkennen ist, lassen sich Modelle definieren, indem die betroffene Klasse von \code{peewee.Model} erbt. Die Verknüpfung zur Datenbank lässt sich durch die Definition des Feldes \code{db} in der Klasse \code{Meta} herstellen. Diese Verknüpfung muss innerhalb einer Klassenhierarchie nur einmal in der obersten Superklasse definiert werden.

\lstset{language=python}
\begin{lstlisting}[caption={Python-Code zur Definition eines \emph{peewee}-Modells. Es modelliert eine Person mit einem Namen und einem Geburtstag. Weiterhin besitzt die Klasse das nicht-persistente Feld \code{age}. \cite{noauthor_quickstart_2023}}, label=code:peewee-model]
class Person(Model):
    class Meta:
        db = SqliteDatabase('people.db')

    name = CharField()
    birthday = DateField()
    age: int
\end{lstlisting}

Der Klasse können dann beliebige Attribute und auch Methoden zugewiesen werden. Soll ein Attribut persistent sein, soll es also in der Datenbank abgelegt werden, so muss ihm in der Klassendefinition ein Feld vom Typ \code{peewee.Field} zugeweisen werden. Über den Subtyp dieses Feldes wird der Typ der korrespondierenden Spalte in der Datenbanktabelle festgelegt. Andere Attribute, wie \code{age} in Quelltext \ref{code:peewee-model} sind Teil des Objektes, aber nicht persistent.

\subsubsection*{Speichern von Objekten}

Objekte lassen sich auf mittels zwei verschiedener Methoden in der Datenbank ablegen. Jede Modell-Klasse erbt die Methode \code{create} von \code{peewee.Model}. Diese nimmt die gleichen Argumente entgegen wie der eigentlich Konstruktor der Klasse. Wird \code{create} aufgerufen, so wird eine Instanz der Klasse erstellt, in die Datenbank geschrieben und im Anschluss zurück gegeben, wie in Quelltext \ref{code:peewee-storing1} ersichtlich wird.

\lstset{language=python}
\begin{lstlisting}[caption={Python-Code zum Erzeugen eines Personen-Objektes und zur Speicherung in der Datenbank in einem Schritt.  \cite{noauthor_quickstart_2023}}, label=code:peewee-storing1]
    uncle_bob = Person.create(name='Bob', birthday=date(1960, 1, 15))
\end{lstlisting}

Alternativ dazu kann das Objekt auch direkt über den Konstruktor erzeugt werden, wie in Quelltext \ref{code:peewee-storing2} zu erkennen. Durch anschließenden Aufruf von \code{save} wird das Objekt dann in die Datenbank gelegt. Diese Methode bietet den Vorteil, dass das Objekt nicht zwangsläufig gespeichert werden muss. Auch erfüllt \code{save} die Funktion der Aktualisierung eines Objektes in der Datenbank, nachdem ein bereits existierendes Objekt verändert wurde.

\lstset{language=python}
\begin{lstlisting}[caption={Python-Code zum Erzeugen eines Personen-Objektes und zur anschließenden Speicherung in der Datenbank.  \cite{noauthor_quickstart_2023}}, label=code:peewee-storing2]
    uncle_bob = Person(name='Bob', birthday=date(1960, 1, 15))
    uncle_bob.save()
\end{lstlisting}

\subsubsection*{Datenabfragen}

Das Abfragen von Daten orientiert sich in \emph{peewee} an den Konzepten von SQL, bildet die SQL-Schlüsselwörter jedoch auch Methoden ab. Quelltext \ref{code:peewee-query} zeigt eine einfach Datenabfrage die zu dem SQL-Statement in Quelltext \ref{code:sql-query} äquivalent ist.

\lstset{language=python}
\begin{lstlisting}[caption={Python-Code zum Abfragen eines Objektes aus der Datenbank.  \cite{noauthor_quickstart_2023}}, label=code:peewee-query]
    grandma = Person.select().where(Person.name == 'Grandma .L').get()
\end{lstlisting}

\lstset{language=sql}
\begin{lstlisting}[caption={SQL-Abfrage mit SELECT und WHERE.}, label=code:sql-query]
    SELECT * FROM Person WHERE name = 'Grandma L.';
\end{lstlisting}
