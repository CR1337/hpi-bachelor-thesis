\chapter{\appendixname}

\section*{Eins (Interview mit Sascha Lesche von der LEAG)}

\begin{description}

    \item[Christian Raue:] Lassen Sie uns zunächst zum Thema des Kohletransportes kommen. Wie viele Tonnen Kohle passen in einen Zug?

    \item[Sascha Lesche:] Ein Zug besteht immer aus einer Lok und 16 Wagen. Jeder Wagen hat eine Aufnahmekapazität von 84 Kubikmeter, was 60 Tonnen Kohle entspricht. Damit kommen Sie auf eine Gesamtmenge von 960 Tonnen.

    \item[Christian Raue:] Welchen Energiegehalt hat die transportierte Kohle im Schnitt?

    \item[Sascha Lesche:] Der Energiegehalt der Kohle schwankt. Da sie nicht gleichmäßig gewachsen ist, weist sie teilweise sehr unterscheidliche chemische Zusammensetzungen auf, welche den Energiegehalt beeinflusst. Für eine Faustformel empfehle ich, sich auf allgemeine Quellen zu berufen.

    \item[Max Lietze:] Als nächstes möchten wir gern mit Ihnen darüber reden, wie wir die Züge in unserer Simulation möglichst realistisch repräsentieren können. Eine Frage ist dabei, wie lang so ein Zug ist.

    \item[Sascha Lesche:] Die Länge eines Wagens beträgt 12,5 Meter. Zusammen mit der Lok kommen Sie dann auf eine Länge von ca. 220 Metern.

    \item[Max Lietze:] Wie schnell kann ein Kohlezug fahren?

    \item[Sascha Lesche:] Die Loks schaffen eine Geschwindigkeit von 60 km/h, dürfen aber gemäß Bau- und Betriebsanweisung maximal 50 km/h fahren.

    \item[Max Lietze:] Wie schnell könne die Züge beschleunigen und abbremsen?

    \item[Sascha Lesche:] Es gibt zunächst sogenannte Zuglastdiagramme, bei welchen die Zugkraft in Zusammenhang mit der Masse gebracht wird. Wahrscheinlich wird es aber schwierig sein, die Werte daraus abzuleiten. Ich kann Ihnen jedoch theoretische Annahmen geben. Für die Beschleunigung wären das 0,15 Mter pro Quadratsekunde und für das Abbremsen 0,4 Meter pro Quadratsekunde.

    \item[Max Lietze:] Wie lange dauert das Be- und Entladen?

    \item[Sascha Lesche:] Das Be- und Entladen dauert zwischen 20 und 30 Minuten. Wenn Sie ihr System stabil laufen lassen wollen, sollten Sie vielleicht besser von 30 Minuten ausgehen.

    \item[Christian Raue:] Wir möchten unsere Simulationsergebnisse gern mithilfe von Echtweltdaten verifizieren. Können Sie uns sagen, in welchem Takt die Züge im Mittel in den Kraftwerken ankommen?

    \item[Sascha Lesche:] Vor vielen Jahren, fuhren die Kraftwerke noch eine reine Grundlast. Da hätte man ein Tagesmittel angeben können. Im Moment ist das nicht möglich. Es gibt je nach Bedarf große Schwankungen. Am Tag können 80 Züge in ein großes Kraftwerk einfahren. Manchmal sind es mehr, bei entsprechend geringeren Leistungen auch weniger.

    \item[Christian Raue:] Wieviele Züge befinden sich im Schnitt gleichzeitig auf dem Schienennetz?

    \item[Sascha Lesche:] Das ist leider genauso schwer zu beantworten. Es können zehn bis zwanzig Züge sein. Bei 25 Zügen ist dann aber langsam ein Limit erreicht, von Reststoff- und Transportfahrten abgesehen.

\end{description}