\section{Ergebnisdiskussion}

Die Ergebnisse in \autoref{fig:results-departures} zeigen, dass die Längen der Abfahrtsperioden der Kohlezüge direkt mit dem berechneten Kohlebedarf korrelieren. Das bedeutet, dass umso mehr Kohlezüge abfahren, je mehr Kohle benötigt wird. Dieses Ergebnis ist sinnvoll und deckt sich mit der Realität und dem was man erwarten würde. Laut Sascha Lesche (siehe Anhang) können an einem Tag etwa 80~Züge in ein großes Kraftwerk einfahren. Diese Zahl kann jedoch je nach tatsächlichem Bedarf auch höher oder niedriger sein. Umgerechnet ergibt sich aus diesem Wert eine mittlere Abfahrtsperiodenlänge von 18~Minuten. Das Ergebnis für das größte der drei Kraftwerke, \emph{Jänschwalde}, liegt mit einer mittleren Abfahrtsperiodenlänge von etwa 19~Minuten sehr nah an diesem Wert. \\
\\
Auch die Verhältnisse der Abfahrtsperiodenlänge der drei Kraftwerke \emph{Jänschwalde}, \emph{Boxberg} und \emph{Schwarze Pumpe} decken sich mit der Realität. Das Kraftwerk \emph{Jänschwalde} hat mit einer Gesamtleistung von rund 3,3~GW die höchste der drei Kraftwerke. Entsprechend muss es den höchsten Kohlebedarf besitzen, was die niedrigste Abfahrtsperiodenlänge von ca. 44~Minuten plausibel macht. Gleiches gilt für das zweit-leistungsfähigste Kraftwerk \emph{Boxberg} mit einer Gesamtleistung von etwa 2,7~GW und auch für das Kraftwerk \emph{Schwarze Pumpe} mit ca. 1,7~GW. \cite{noauthor_bundesnetzagentur_nodate}\\
\\
Zusammenfassen lässt sich sagen, dass die Ergebnisse der Simulation plausibel sind und sich mit der Realität decken.