% => Wenn die Arbeit auf Deutsch verfasst wurde, verlangt das Studienreferat KEINEN englischen Abstract

% % englischer Abstract
%\null\vfil
%\begin{otherlanguage}{english}
%\begin{center}\textsf{\textbf{\abstractname}}\end{center}
%
%\noindent Lorem ipsum dolor sit amet, consetetur sadipscing elitr, sed diam nonumy eirmod tempor invidunt ut labore et dolore magna aliquyam erat, sed diam voluptua. At vero eos et accusam et justo duo dolores et ea rebum. Stet clita kasd gubergren, no sea takimata sanctus est Lorem ipsum dolor sit amet. Lorem ipsum dolor sit amet, consetetur sadipscing elitr, sed diam nonumy eirmod tempor invidunt ut labore et dolore magna aliquyam erat, sed diam voluptua. At vero eos et accusam et justo duo dolores et ea rebum. Stet clita kasd gubergren, no sea takimata sanctus est Lorem ipsum dolor sit amet.
%
%\end{otherlanguage}
%\vfil\null


% => Wenn die Arbeit auf Englisch verfasst wurde, verlangt das Studienreferat einen englischen UND deutschen Abstract (der dt. Abstract kann dann ggf. auch ans Ende der Arbeit)

% deutsche Zusammenfassung
\null\vfil
\begin{otherlanguage}{ngerman}
\begin{center}\textsf{\textbf{\abstractname}}\end{center}

\noindent Das Projekt \emph{FlexiDug} erforscht die Nachnutzungsmöglichkeiten der Schieneninfrastruktur, die bisher dem Braunkohletransport in der Lausitz diente. Ziel ist es, ein Konzept für eine gemeinsame Nutzung für Personen- und Güterzüge zu erstellen. Diese Arbeit ist Teil eines Softwareprojekts, das die Simulation des Zugverkehrs auf Basis einer Erweiterung des Verkehrssimulators \emph{SUMO} ermöglicht. Die Arbeit beschäftigt sich mit der Erzeugung von Zügen innerhalb der Simulation. Dafür werden drei verschiedene Arten von Abfahrtsplänen benötigt: regelmäßige, zufällige und bedarfsorientierte. Bedarsforientierte Abfahrtspläne basieren auf dem berechneten Kohlebedarf von Kraftwerken. Die Berechnung basiert auf Stromproduktionsdaten der Bundesntzagentur. Die Ergebnisse der Simulation zeigen, dass die Abfahrtsperioden der Kohlezüge mit dem Kohlebedarf korrelieren und dass die Software einen realistischen Zugverkehr simulieren kann.

\end{otherlanguage}
\vfil\null



