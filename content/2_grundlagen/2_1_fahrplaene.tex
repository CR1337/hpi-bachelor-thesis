\section{Fahrpläne}

Ein Fahrplan ist die zeitliche und räumliche Festlegung von Zugfahrten auf einem Schienennetz \cite{groger_simulation_2002}. Er enthält unter anderem
\begin{itemize}
    \item die Zuggattung,
    \item die Verkehrstage,
    \item die Strecke,
    \item die Ankunfts- Durchfahr- und Abfahrzeiten in allen Halte- und Betriebsstellen,
    \item und die zulässigen Geschwindigkeiten in den einzelnen Streckenabschnitten.
\end{itemize}
Der Fahrplan dient einer Reihe von Zwecken. Zum einen hilft der Fahrplan dabei, die vorhandene Infrastruktur auf die verkehrenden Züge zu verteilen und den Verkehr somit zu koordinieren. Damit können mögliche Konflikte bereits bei der Erstellung des Fahrplans lokalisiert werden. Zum anderen dient der Fahrplan als Informationsquelle für den Kunden und hilft diesem bei der Entscheidungsfindung bezüglich der Wahl der geeigneten Verkehrsmittel. Er dient den Triebfahrzeugführern und Fahrdienstleitern als Grundlage für ihre Betriebsplanung und ermöglicht Aussagen über benötigte Ressourcen. So bedeutet ein hoher Takt beispielsweise auch eine hohe Anzahl an benötigten Fahrzeugen und Personal. Zusätzlich ermöglicht ein Fahrplan das Treffen von Aussagen über die Leistungsfähigkeit des Zugverkehrs. \cite{groger_simulation_2002}
