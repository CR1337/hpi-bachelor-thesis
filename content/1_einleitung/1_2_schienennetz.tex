\section{Das Schienennetz der LEAG}
\todo{Noch in Arbeit}
Essentiell für den Betrieb der Kraftwerke ist die regelmäßige und zuverlässige Belieferung mit Braunkohle. Zu diesem Zweck betreibt die LEAG  ein Schienennetz von 391 Kilometern Länge. Es verbindet die Tagebaue Jänschwalde, Welzow-Süd, Nochten und Reichwalde mit den Braunkohlekraftwerken Jänschwalde, Boxberg und Schwarze Pumpe. Außerdem ist der Kohleveredelungsbetrieb in Schwarze Pumpe angeschlossen. Gleichzeitig fahren bis zu 25 Kohlezüge auf dem Netz. Sie dienen nicht nur allein dem Transport der Kohle. Ebenso befördern sie die Abfallprodukte der Kohleverstromung, Asche und Gips. Mit ca. 1600 Tonnen pro Zug erreichen sie eine Maximalgeschwindigkeit von 50 km/h und einen Bremsweg von 400 Metern. Zum Fuhrpark gehören 61 E-Loks und 14 Diesel-Loks.

\begin{figure}[H]
	\centering
	\includegraphics[width=0.75\linewidth]{images/LEAG-Netz-annotated.png}
	\caption{LEAG Netz}
	\label{fig:leag-netz}
\end{figure}