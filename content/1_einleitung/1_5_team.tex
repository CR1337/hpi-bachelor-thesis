\section{Das Softwareprojekt und die Interaktion im Team}

Dieser Abschnitt beschreibt die einzelnen Arbeitspakete des Softwareprojektes und wie die Verantwortlichkeiten innerhalb des Teams verteilt wurden. Weiterhin wird auf gemeinsame Teamarbeit eingegangen.

\subsection{Arbeitspakete und Verantwortlichkeiten}

Das Team besteht aus sechs Personen. Entsprechend wurde das Softwareprojekt in sechs Arbeitspakete geteilt, welche dann getrennt bearbeitet werden konnten. Die zugehörigen Arbeiten und die Pakete, mit denen sie sich befassen sind in \autoref{tab:team-components} dargestellt.

\begin{table}[!ht]
	\centering
	\caption{Die Arbeitspakete dieses Projektes, die Arbeiten, welche sich mit diesen beschäftigen und die Autoren dieser Arbeiten}
	\label{tab:team-components}
	\begin{tabular}{|l|m{0.5\linewidth}|l|}
		\hline
		\textbf{Arbeitspaket} & \textbf{Titel der Arbeit} & \textbf{Autor} \\
		\hline
		\hline
		\emph{REST-API} & \citetitle{kamp_architektur_2023} \cite{kamp_architektur_2023} & \citeauthor{kamp_architektur_2023} \\
		\hline
		die Fehlerinjektion & \citetitle{persitzky_fehlerinjektion_2023} \cite{persitzky_fehlerinjektion_2023} & \citeauthor{persitzky_fehlerinjektion_2023} \\
		\hline
        die Routenberechnung & \citetitle{lietze_evaluierung_2023} \cite{lietze_evaluierung_2023} & \citeauthor{lietze_evaluierung_2023} \\
		\hline
        die \emph{SUMO}-Schnittstelle & \citetitle{ortlam_implementierung_2023} \cite{ortlam_implementierung_2023} & \citeauthor{ortlam_implementierung_2023} \\
		\hline
        die Datenauswertung & \citetitle{reisener_entwurf_2023} \cite{reisener_entwurf_2023} & \citeauthor{reisener_entwurf_2023} \\
		\hline
        die Zugerzeugung & diese Arbeit &  \\
		\hline
	\end{tabular}
\end{table}

Die \emph{REST-API} stellt eine Benutzerschnittstelle für die von uns entwickelte Simulationsumgebung bereit. Die entsprechende Arbeit beschäftigt sich außerdem mit der komponenten-übergreifenden Architektur und der Interaktion der einzelnen Softwarekomponenten. Die Fehlerinjektion dient der Erzeugung von Fehlern, welche der Simulation von unvorhergesehenen und unerwünschten Ereignissen im Rahmen des Bahnverkehrs dienen. Die Routenberechnung plant und steuert die Bewegung der Züge durch das Schienennetz. Sie interagiert mit der Stellwerkslogik und bedient die Schnittstelle zu \emph{SUMO}. Die \emph{SUMO}-Schnittstelle ist ein objektorientierter \emph{Wrapper}\footnote{Ein \emph{Wrapper} dient der Bereitstellung einer anderen Schnittstelle für bereits existierenden Funktionalitäten.}, welcher die für uns wichtigen Aspekte der Schnittstelle zu \emph{SUMO} kapselt und für uns die Arbeit mit ihnen erleichtert. Die Datenauswertung stellt die Ergebnisse der Simulation grafisch dar und ermöglicht die Analyse der Simulationsergebnisse. Die Zugerzeugung wird in dieser Arbeit ausführlich behandelt.

\subsection{Die Teamarbeit}

Um unser Projekt effizient und flexibel zu gestalten, haben wir agile Arbeitsmethoden angewendet. Dabei haben wir uns auf selbstbestimmtes Arbeiten und Gleichberechtigung aller Teammitglieder fokussiert. Um unsere Arbeit zu organisieren und zu koordinieren, haben wir verschiedene Werkzeuge genutzt. Das wichtigste davon war das GitHub-Repositorium\footnote{url{https://github.com/BP2022-AP1/bp2022-ap1}}, das uns die Quelltextverwaltung ermöglicht hat. Außerdem haben wir über GitHub ein Wiki als Wissenssammlung und ein Kanban-Board für die Verwaltung unserer Aufgaben erstellt. Das Kanban-Board hat uns geholfen, unsere Aufgaben zu visualisieren und transparent zu machen. Wir haben auch die Anzahl der gleichzeitig zu bearbeitenden Aufgaben begrenzt und unsere Arbeitsabläufe iterativ verbessert. Die Kommunikation haben wir durch die Kollaborationsplattform Slack und durch morgendliche Meetings gewährleistet.